\subsection{%
  Лекция \texttt{24.??.??}.%
}

Цель существования современной операционной системы заключается в том, что она
должна обеспечить производительность, надежность и безопасность выполнения
пользовательских программ, эксплуатации аппаратного обеспечения, хранения и
доступа к данным (в том числе по сети) и диалога с пользователем.

Т.к. операционная система это сложное и комплексное ПО, то чтобы лучше его
понять, сначала надо поговорить об его архитектуре. Выделяют несколько уровней
архитектуры программного обеспечения:

\begin{enumerate}
\item
  Функциональная архитектура.
  
  Этот уровень описывает всю совокупность функций, выполняемых операционной
  системой.

\item
  Системная архитектура.

  Реализация операционной системы это программно-аппаратный комплекс: есть
  аппаратные компоненты, которые являются неотъемлемой частью операционной
  системы, а есть программные компоненты, которые могут быть как свободными, так
  и проприетарными библиотеками. 

\item
  Программная архитектура.

  Т.к. операционная система содержит некоторые программные компоненты, то нужно
  учитывать, что эти компоненты также обладают собственной структурой.

\item
  Архитектура данных.

  Предыдущий пункт говорит нам о сложности организации кода, а значит этот код
  работает с не менее сложноорганизованными данными, поэтому имеет смысл
  говорить об архитектуре данных.
\end{enumerate}

\subheader{Функциональная архитектура. Функции операционной системы}

\begin{enumerate}
\item
  Управление разработкой и исполнением пользовательского ПО.

  \begin{enumerate}
  \item
    Предоставление возможности и API для написания программного обеспечения,
    совместимого с этой операционной системой.
  
  \item
    Предоставление возможности загрузить и выполнить написанное приложение, а
    также обеспечить ему доступ к требуемым в процессе исполнения ресурсам.
  
  \item
    Обнаружение и обработка ошибок, возникающих в ходе выполнения написанного
    ПО.

  \item
    Высокоуровневый доступ к устройствам ввода-вывода.
  
  \item
    Управление хранилищем данных: обеспечение высокоуровневого доступа к данным
    и их безопасности.

  \item
    Мониторинг ресурсов.
  \end{enumerate}

\item
  Оптимизация использования ресурсов.
  
  Обозначим \(k_1, k_2, \dotsc\)~--- критерии оптимальности использования
  соответствующего ресурса. Т.к. в распоряжении вычислительного узла обычно
  находится не один, а множество ресурсов, то чаще всего оптимизировать все
  критерии одновременно невозможно. Из-за этого операционные системы используют
  разные принципы оптимизации, например:

  \begin{enumerate}
  \item
    Суперкритерий (свертка).

    Пусть \(\widehat{k} = \alpha k_1 + \beta k_2 + \gamma k_3 + \dotsc\) при
    условии, что \(\alpha + \beta + \gamma + \dotsc = 1\), т.е. используется
    взвешенная сумма критериев. Выбирается та стратегия, у которой суперкритерий
    \(\widehat{k}\) максимален.
  
  \item
    Условный критерий.

    В некоторых ситуациях значения какого-либо критерия (или нескольких
    критериев) обязательно должны находится в некотором диапазоне. Тогда сначала
    ищется \quote{область} в которой выполнены требуемые условия к критериям, а
    потом уже оптимизируются остальные критерии.
  \end{enumerate}
  
  Стоит помнить, что операционная система это открытая система: пользователи
  открывают новые приложения, закрывают старые, запускают на обработку большие
  объемы данных или наоборот, бездействуют~--- в общем, помимо того, что
  требуется решать сложную задачу многокритериальной оптимизации, также нужно
  учитывать текущий контекст. Операционные системы обычно для решения этой
  проблемы используют ту или иную реализацию цикла Деминга (PDCA). Данный цикл
  состоит из четырех этапов:

  \begin{enumerate}
  \item
    Планирование. На этом этапе формируются некоторые значения коэффициентов
    \(\alpha, \beta, \gamma, \dotsc\).

  \item
    Выполнение. На этом этапе операционная система принимает решения согласно
    выбранному плану.

  \item
    Проверка. На данном этапе происходит проверка сделанных решений на
    соответствие некоторым целевым показателям.

  \item
    Действия. На данном этапе нужно каким либо образом исправить несоответствие
    полученного результата плану.
  \end{enumerate}

  Далее система уходит на новый цикл: на новом этапе планирования может быть
  выбрана другая стратегия, т.к. ситуация поменялась, или та же самая, если она
  хорошо себя зарекомендовала.

\item
  Поддержка администрирования и эксплуатации вычислительного узла.

  Операционная система должна предоставлять средства для диагностики, системного
  администрирования (восстановление после сбоев), восстановления поврежденных
  файлов (резервное копирование).

\item
  Поддержка развития самой операционной системы.
  
  Операционные системы, как чрезвычайно сложное ПО, проектируются и создаются в
  течение очень долго времени, а значит и использоваться будут также длительное
  время (процесс перехода на новую ОС связан с риском и определенными
  затратами). Это значит, что ОС должна быть открыта к изменениям: за время ее
  использования появятся новые программные и аппаратные решения, будет написано
  новое ПО, и ОС должна быть спроектирована так, чтобы у этих продуктов была
  возможность работать в рамках этой ОС (либо должна быть возможность добавить
  их поддержку). В современных ОС эта функция обычно реализуется с помощью
  средств автоматического обновления.
\end{enumerate}

\subheader{Функциональные подсистемы}

\begin{enumerate}
\item
  Подсистема управления процессами.

  \begin{enumerate}
  \item
    Планировщики.

  \item
    Структуры данных, отвечающие за хранение данных о процессах.
  \end{enumerate}

\item
  Подсистема управления памятью.

  \begin{enumerate}
  \item
    Механизм виртуализации памяти.

  \item
    Защита памяти одного процесса от других процессов.

  \item
    Распределение данных по памяти с разной скоростью доступа. 
  \end{enumerate}

\item
  Подсистема управления файлами.

  \begin{enumerate}
  \item
    Преобразование символьных имен файлов в адреса их физического хранения.

  \item
    Механизм управления каталогами.
  \end{enumerate}

\item
  Подсистема управления внешними устройствами.

\item
  Подсистема защиты данных и администрирования.

  \begin{enumerate}
  \item
    Идентификация, аутентификация, авторизация пользователя.

  \item
    Аудит операционной системы, действий пользователя, поведения приложений,
    сетевой активности и т.д.
  \end{enumerate}

\item
  API.

\item
  Подсистема пользовательского интерфейса.

  \begin{enumerate}
  \item
    Интерфейс командной строки.

  \item
    Графический пользовательский интерфейс (GUI).
  \end{enumerate}
\end{enumerate}
