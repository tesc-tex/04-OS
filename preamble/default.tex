\newcommand{\todo}{\textbf{\color{red} TODO: \hspace{6pt}}}
\newcommand{\LeC}{\hbox{L\kern-.125em\lower.5ex\hbox{E}\kern-.075em C}}

\renewcommand{\dateseparator}{.}
\setlength\parindent{0pt}
\DeclareSymbolFontAlphabet{\mathbbold}{bbold}
\DeclareSymbolFontAlphabet{\mathbb}{AMSb}

\let\pair\relax
\let\abs\relax
\newcommand{\pair}[1]{\left\langle #1 \right\rangle}
\newcommand{\set}[1]{\left\{ #1 \right\}}
\newcommand{\abs}[1]{\left\lvert #1 \right\rvert}
\NewDocumentCommand{\prh}{o m}{%
  \IfNoValueTF{#1}%
  { \left( #2 \right) }%
  { #1( #2 #1) }%
}
\newcommand{\given}[1][\Big]{\nonscript\,#1\vert\nonscript\,\mathopen{}}

\renewcommand{\phi}{\varphi}
\renewcommand{\epsilon}{\varepsilon}
\newcommand{\NN}{\mathbb{N}}
\newcommand{\ZZ}{\mathbb{Z}}
\newcommand{\RR}{\mathbb{R}}
\newcommand{\CC}{\mathbb{C}}

\renewcommand{\bar}{\overline}
\renewcommand{\vec}[1]{\vv{\vphantom{b}#1}}
\renewcommand{\ge}{\geqslant}
\renewcommand{\le}{\leqslant}

\newcommand{\display}[1]{\displaystyle{#1}}
\newcommand{\existsone}{\exists!\,}
\renewcommand{\quote}[1]{\enquote{#1}}

\renewcommand{\iff}[1][]{\xLongleftrightarrow{\text{#1}}}
\renewcommand{\implies}[1][]{\xLongrightarrow{\text{#1}}}
\renewcommand{\impliedby}[1][]{\xLongleftarrow{\text{#1}}}
\newcommand{\bydef}{\iff[def]}
\newcommand{\eqby}[1]{\xlongequal{\text{#1}}}

\newcommand{\ness}{\((\impliedby)\) }
\newcommand{\suff}{\((\implies)\) }

\newcommand{\larr}[1]{\xleftarrow{#1}}
\newcommand{\rarr}[1]{\xrightarrow{#1}}
\newcommand{\Larr}[1]{\xlongleftarrow{#1}}
\newcommand{\Rarr}[1]{\xlongrightarrow{#1}}

\newcommand{\mtx}[1]{\begin{matrix} #1 \end{matrix}}
\newcommand{\mtxp}[1]{\begin{pmatrix} #1 \end{pmatrix}}
\newcommand{\mtxv}[1]{\begin{vmatrix} #1 \end{vmatrix}}
\newcommand{\mtxb}[1]{\begin{bmatrix} #1 \end{bmatrix}}

\newcommand{\makesubfigure}[4]{
  \begin{subfigure}[b]{#4\textwidth}
    \centering
    \input{figures/#1#2.tex}
    \caption{#3}
    \label{fig:#1#2}
  \end{subfigure}
}
\newcommand{\galleryone}[3][h]{
  \refstepcounter{points}
  \begin{figure}[#1]
    \centering
    \input{figures/#2.tex}
    \caption{#3}
    \label{fig:#2}
  \end{figure}
}
\newcommand{\gallerytwo}[4]{
  \refstepcounter{points}
  \begin{figure}[h]
    \centering
    \makesubfigure{#1}{a}{#3}{0.48}
    \makesubfigure{#1}{b}{#4}{0.48}
    \caption{#2}
    \label{fig:#1}
  \end{figure}
}
\newcommand{\gallerydouble}[4]{
  \begin{minipage}{\textwidth}
    \centering
    \begin{minipage}{0.45\textwidth}
      \galleryone[H]{#1}{#2}
    \end{minipage}%
    \hspace{0.05\textwidth}%
    \begin{minipage}{0.45\textwidth}
      \galleryone[H]{#3}{#4}
    \end{minipage}
  \end{minipage}
}
\newcommand{\gallerythree}[5]{
  \refstepcounter{points}
  \begin{figure}[h]
    \centering
    \makesubfigure{#1}{a}{#3}{0.3}%
    \hspace{0.03\textwidth}%
    \makesubfigure{#1}{b}{#4}{0.3}%
    \hspace{0.03\textwidth}%
    \makesubfigure{#1}{c}{#5}{0.3}
    \caption{#2}
    \label{fig:#1}
  \end{figure}
}
\newcommand{\figref}[1]{рис.~\ref{fig:#1}}

\setlength{\columnsep}{1cm}
\newenvironment{twocolumns}{
  \begin{minipage}{\linewidth}
    \begin{multicols}{2}
}{
    \end{multicols}
  \end{minipage}
  \medskip
}

\newenvironment{important}{
  \begin{tcolorbox}[
    colback = blue!5!white,
    colframe = blue!75!black,
    boxsep = 12pt,
    left = 0pt,
    right = 0pt,
    top = 0pt,
    bottom = 0pt,
  ]
}{
  \end{tcolorbox}
}

\makeatletter
\newcommand\underparen[1]{\@ifnextchar_{\uphelp{\uparen{#1}}}{\uparen{#1}}}
\makeatother
\def\uphelp#1_#2{\ensurestackMath{\stackunder[1pt]{#1}{\scriptstyle #2}}}
\newcommand\uparen[1]{\setbox0=\hbox{$#1$}\ensurestackMath{%
  \stackunder[0pt]{#1}{\rotatebox{90}{$\left(%
  \rule[\dimexpr-.5\wd0+\dp\strutbox-1.3pt]{0pt}{\wd0}\right.$}}%
}}
\newcommand{\under}[2]{\underparen{#1}_{#2}}

\sectionfont{\color{blue}}
\subsectionfont{\color{blue}}

\renewcommand{\thesection}{\arabic{section}.}
\renewcommand{\thesubsection}{\thesection\arabic{subsection}.}

\newcounter{points}[subsection]
\renewcommand{\thepoints}{\thesubsection\arabic{points}}

\declaretheoremstyle[
  headfont = \normalfont\bfseries,
  notefont = \normalfont\bfseries,
  bodyfont = \normalfont,
  headpunct = {},
  headformat={\NAME~\NUMBER.\NOTE},
]{theoremstyle}

\declaretheoremstyle[
  spaceabove = 6pt,
  spacebelow = 6pt,
  headfont = \normalfont\itshape,
  notefont = \normalfont,
  bodyfont = \normalfont,
  postheadspace = 3pt,
]{remarkstyle}

\declaretheoremstyle[
  spaceabove = 6pt,
  spacebelow = 6pt,
  headfont = \normalfont\bfseries,
  notefont = \normalfont,
  bodyfont = \normalfont,
  postheadspace = 3pt,
]{definitionstyle}

\declaretheorem[
  title = Теорема,
  sibling = points,
  style = theoremstyle
]{theorem}

\declaretheorem[
  title = Замечание,
  sibling = points,
  style = remarkstyle
]{remark}

\declaretheorem[
  title = Следствие,
  sibling = points,
  style = remarkstyle
]{corollary}

\declaretheorem[
  title = Пример,
  sibling = points,
  style = remarkstyle
]{example}

\declaretheorem[
  title = Def,
  sibling = points,
  style = definitionstyle
]{definition}

\declaretheorem[
  title = \underline{Lm},
  sibling = points,
  style = definitionstyle
]{lemma}

\addto\captionsrussian{\renewcommand{\proofname}{\(\square\)\nopunct}}
\renewcommand{\qedsymbol}{\(\blacksquare\)}

\addto\captionsrussian{\renewcommand{\figurename}{Рис.}}
\renewcommand{\thefigure}{\arabic{section}.\arabic{subsection}.\arabic{points}}

\newcommand{\subheader}[1]{\medskip \textbf{#1} \medskip}
\newcommand{\subsubheader}[2]{\medskip \textbf{#1} \vspace{6pt} #2 \smallskip}

\pagestyle{fancy}
\fancyhead{}
\fancyfoot{}
\fancyfoot[C]{\thepage/\pageref*{LastPage}}
\fancypagestyle{mainpage}{
  \fancyfoot{}
  \fancyfoot[C]{\url{https://github.com/isagila/tesc}}
}
\renewcommand{\headrulewidth}{0pt}

\setlist[enumerate]{ topsep = 0pt }
